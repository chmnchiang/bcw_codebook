\documentclass[10pt,twocolumn,oneside]{article}
\setlength{\columnsep}{10pt}                                                                    %兩欄模式的間距
\setlength{\columnseprule}{0pt}                                                                %兩欄模式間格線粗細

\usepackage{amsthm}								%定義,例題
\usepackage{amssymb}
%\usepackage[margin=2cm]{geometry}
\usepackage{color}
\usepackage[x11names]{xcolor}
\usepackage{xeCJK}								%xeCJK
\usepackage{listings}								%顯示code用的
%\usepackage[Glenn]{fncychap}						%排版,頁面模板
\usepackage{fancyhdr}								%設定頁首頁尾
\usepackage{graphicx}								%Graphic
\usepackage{enumerate}
\usepackage{titlesec}
\usepackage{amsmath}
\usepackage{fontspec}								%設定字體

%\usepackage[T1]{fontenc}
\usepackage{amsmath, courier, listings, fancyhdr, graphicx}
\topmargin=0pt
\headsep=5pt
\textheight=780pt
\footskip=0pt
\voffset=-40pt
\textwidth=545pt
\marginparsep=0pt
\marginparwidth=0pt
\marginparpush=0pt
\oddsidemargin=0pt
\evensidemargin=0pt
\hoffset=-42pt

%\renewcommand\listfigurename{圖目錄}
%\renewcommand\listtablename{表目錄} 

%%%%%%%%%%%%%%%%%%%%%%%%%%%%%

% \setmainfont{Consolas}				%主要字型
% \setCJKmainfont{Consolas}			%中文字型
\setmainfont{Linux Libertine O}
\setmonofont{Source Code Pro}
\XeTeXlinebreaklocale "zh"						%中文自動換行
\XeTeXlinebreakskip = 0pt plus 1pt				%設定段落之間的距離
\setcounter{secnumdepth}{3}						%目錄顯示第三層

%%%%%%%%%%%%%%%%%%%%%%%%%%%%%
\makeatletter
\lst@CCPutMacro\lst@ProcessOther {"2D}{\lst@ttfamily{-{}}{-{}}}
\@empty\z@\@empty
\makeatother
\lstset{											% Code顯示
language=C++,										% the language of the code
basicstyle=\footnotesize\ttfamily, 						% the size of the fonts that are used for the code
%numbers=left,										% where to put the line-numbers
numberstyle=\footnotesize,						% the size of the fonts that are used for the line-numbers
stepnumber=1,										% the step between two line-numbers. If it's 1, each line  will be numbered
numbersep=5pt,										% how far the line-numbers are from the code
backgroundcolor=\color{white},					% choose the background color. You must add \usepackage{color}
showspaces=false,									% show spaces adding particular underscores
showstringspaces=false,							% underline spaces within strings
showtabs=false,									% show tabs within strings adding particular underscores
frame=false,											% adds a frame around the code
tabsize=2,											% sets default tabsize to 2 spaces
captionpos=b,										% sets the caption-position to bottom
breaklines=true,									% sets automatic line breaking
breakatwhitespace=false,							% sets if automatic breaks should only happen at whitespace
escapeinside={\%*}{*)},							% if you want to add a comment within your code
morekeywords={*},									% if you want to add more keywords to the set
keywordstyle=\bfseries\color{Blue1},
commentstyle=\itshape\color{Red4},
stringstyle=\itshape\color{Green4},
}

%%%%%%%%%%%%%%%%%%%%%%%%%%%%%

\begin{document}
\pagestyle{fancy}
\fancyfoot{}
%\fancyfoot[R]{\includegraphics[width=20pt]{ironwood.jpg}}
\fancyhead[L]{National Taiwan University bcw0x1bd2}
\fancyhead[R]{\thepage}
\renewcommand{\headrulewidth}{0.4pt}
\renewcommand{\contentsname}{Contents} 

\scriptsize
\tableofcontents
%%%%%%%%%%%%%%%%%%%%%%%%%%%%%

\newpage

\section{Basic}
\subsection{Increase Stack}
\lstinputlisting{../codes/Basic/increase_stack/increase_stack.cpp}
\subsection{Default}
\lstinputlisting{../codes/Basic/default/default.cpp}
\section{Data Structure}
\subsection{Heavy Light Decomposition}
\lstinputlisting{../codes/Data_Structure/Heavy_Light_Decomposition/heavy_light_decomposition.cpp}
\subsection{Bigint}
\lstinputlisting{../codes/Data_Structure/Bigint/Bigint.cpp}
\subsection{Leftist Heap}
\lstinputlisting{../codes/Data_Structure/Leftist_Heap/Leftist_Heap.cpp}
\subsection{Treap}
\lstinputlisting{../codes/Data_Structure/Balance_Tree/treap/treap.cpp}
\subsection{Unordered Hash Function}
\lstinputlisting{../codes/Data_Structure/Balance_Tree/unordered_hash_function/unordered_hash_function.cpp}
\section{Graph}
\subsection{Tarjan}
\lstinputlisting{../codes/Graph/Tarjan/Tarjan.cpp}
\subsection{Kosaraju Scc}
\lstinputlisting{../codes/Graph/kosaraju_scc/kosaraju_scc.cpp}
\subsection{SW Min Cut}
\lstinputlisting{../codes/Graph/Flow/SW_min_cut/SW_min_cut.cpp}
\subsection{Dinic}
\lstinputlisting{../codes/Graph/Flow/dinic/dinic.cpp}
\subsection{Cost Flow}
\lstinputlisting{../codes/Graph/Flow/cost_flow/cost_flow.cpp}
\subsection{Isap}
\lstinputlisting{../codes/Graph/Flow/isap/isap.cpp}
\subsection{2 Com Flow}
\lstinputlisting{../codes/Graph/Flow/2_com_flow/2_com_flow.cpp}
\subsection{DMST With Sol}
\lstinputlisting{../codes/Graph/DMST_with_sol/dmst_with_sol.cpp}
\subsection{Maximum Clique}
\lstinputlisting{../codes/Graph/Maximum_Clique/Maximum_Clique.cpp}
\subsection{Minimum Mean Cycle}
\lstinputlisting{../codes/Graph/minimum_mean_cycle/minimum_mean_cycle.cpp}
\subsection{Kuhn Munkress}
\lstinputlisting{../codes/Graph/Matching/Kuhn_Munkress/Kuhn_Munkress.cpp}
\subsection{Simple Graph Matching}
\lstinputlisting{../codes/Graph/Matching/simple_graph_matching/simple_graph_matching.cpp}
\subsection{Bipartite Matching}
\lstinputlisting{../codes/Graph/Matching/bipartite_matching/bipartite_matching.cpp}
\section{Math}
\subsection{Simplex}
\lstinputlisting{../codes/Math/simplex/simplex.cpp}
\subsection{Ax+by=gcd}
\lstinputlisting{../codes/Math/ax+by=gcd/ax+by=gcd.cpp}
\subsection{Fast Fourier Transform}
\lstinputlisting{../codes/Math/fast_fourier_transform/Fast_Fourier_Transform.cpp}
\subsection{Pollard Rho}
\lstinputlisting{../codes/Math/pollard_rho/pollard_rho.cpp}
\subsection{Polygen}
\lstinputlisting{../codes/Math/polygen/polygen.cpp}
\subsection{Gauss Elimination}
\lstinputlisting{../codes/Math/gauss_elimination/Gauss_Elimination.cpp}
\subsection{Mod Utils}
\lstinputlisting{../codes/Math/mod_utils/mod_utils.cpp}
\subsection{Ntt}
\lstinputlisting{../codes/Math/ntt/ntt.cpp}
\subsection{Chinese Remainder}
\lstinputlisting{../codes/Math/chinese_remainder/Chinese_Remainder.cpp}
\subsection{Primes}
\lstinputlisting{../codes/Math/primes/primes.cpp}
\subsection{Miller Rabin}
\lstinputlisting{../codes/Math/miller_rabin/miller_rabin.cpp}
\section{Geometry}
\subsection{Half Plane Intersection}
\lstinputlisting{../codes/Geometry/Half_plane_intersection/Half_plane_intersection.cpp}
\subsection{KD Tree}
\lstinputlisting{../codes/Geometry/KD_tree/KD_Tree.cpp}
\subsection{Minimum Covering Circle}
\lstinputlisting{../codes/Geometry/Minimum_covering_circle/minimum_covering_circle.cpp}
\subsection{Minkowski Sum}
\lstinputlisting{../codes/Geometry/Minkowski_Sum/Minkowski_Sum.cpp}
\subsection{Convex Hull}
\lstinputlisting{../codes/Geometry/Convex_Hull/convex_hull.cpp}
\subsection{Intersection Of Two Circles}
\lstinputlisting{../codes/Geometry/Intersection_of_two_circles/Intersection_of_two_circles.cpp}
\subsection{Intersection Of Two Lines}
\lstinputlisting{../codes/Geometry/Intersection_of_two_lines/Intersection_of_two_lines.cpp}
\subsection{Point Operators}
\lstinputlisting{../codes/Geometry/Point_operators/Point_operators.cpp}
\section{Stringology}
\subsection{AC Automata}
\lstinputlisting{../codes/Stringology/Automata/AC_Automata/AC_Automata.cpp}
\subsection{SAM}
\lstinputlisting{../codes/Stringology/Automata/SAM/SAM.cpp}
\subsection{Palindrome}
\lstinputlisting{../codes/Stringology/Z_Value/palindrome/palindrome.cpp}
\subsection{Z Value}
\lstinputlisting{../codes/Stringology/Z_Value/z_value/z_value.cpp}
\subsection{Smallest Rotation}
\lstinputlisting{../codes/Stringology/Smallest_Rotation/smallest_rotation.cpp}
\subsection{Suffix Array}
\lstinputlisting{../codes/Stringology/Suffix_Array/suffix_array/Suffix_Array.cpp}
\subsection{Sais}
\lstinputlisting{../codes/Stringology/Suffix_Array/sais/sais.cpp}
\section{Problems}
\subsection{Orange Protection}
\lstinputlisting{../codes/Problems/orange_protection/orange_protection.cpp}
\subsection{Max Tangent}
\lstinputlisting{../codes/Problems/max_tangent/max_tangent.cpp}
\subsection{Cot4}
\lstinputlisting{../codes/Problems/cot4/cot4.cpp}


\end{document}
